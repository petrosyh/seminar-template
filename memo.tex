%-----------------------------------------------------------------------------
%
%               Template for sigplanconf LaTeX Class
%
% Name:         sigplanconf-template.tex
%
% Purpose:      A template for sigplanconf.cls, which is a LaTeX 2e class
%               file for SIGPLAN conference proceedings.
%
% Guide:        Refer to "Author's Guide to the ACM SIGPLAN Class,"
%               sigplanconf-guide.pdf
%
% Author:       Paul C. Anagnostopoulos
%               Windfall Software
%               978 371-2316
%               paul@windfall.com
%
% Created:      15 February 2005
%
%-----------------------------------------------------------------------------


\documentclass[nocopyrightspace]{sigplanconf}

% The following \documentclass options may be useful:

% preprint      Remove this option only once the paper is in final form.
% 10pt          To set in 10-point type instead of 9-point.
% 11pt          To set in 11-point type instead of 9-point.
% authoryear    To obtain author/year citation style instead of numeric.

\usepackage{amsmath}


\begin{document}

\special{papersize=8.5in,11in}
\setlength{\pdfpageheight}{\paperheight}
\setlength{\pdfpagewidth}{\paperwidth}

\conferenceinfo{CONF 'yy}{Month d--d, 20yy, City, ST, Country} 
\copyrightyear{20yy} 
\copyrightdata{978-1-nnnn-nnnn-n/yy/mm} 
\doi{nnnnnnn.nnnnnnn}

% Uncomment one of the following two, if you are not going for the 
% traditional copyright transfer agreement.

%\exclusivelicense                % ACM gets exclusive license to publish, 
                                  % you retain copyright

%\permissiontopublish             % ACM gets nonexclusive license to publish
                                  % (paid open-access papers, 
                                  % short abstracts)

\titlebanner{banner above paper title}        % These are ignored unless
\preprintfooter{short description of paper}   % 'preprint' option specified.

\title{CompCert with integer-pointer casting}
\subtitle{}

\authorinfo{yonghyun Kim}
           {Seoul National University}
           {yonghyun.kim@sf.snu.ac.kr}

\maketitle

\begin{abstract}
The certified C compiler CompCert uses an abstract memory model which allows for many optimizations, but in which the behavior of integer-pointer casts is undefined. In [1], Kang et al. present a new formal memory model named "Quasi-concrete model" that supports integer-pointer casts semantics, while still allowing common optimizations. In this talk, I am going to introduce CompCert with Quasi-concrete model so-called "CompCert-Intptr". I will present several concepts: (1) Quasi-Concrete model (2) Mixed Simulation.
\end{abstract}

\section{Quasi-concrete model}
At first, I will remind you about "Quasi-Concrete model" that introduced in [1]. It is a hybrid of Logical and Concrete Memory model. In this model, memory blocks can be either concrete like flat memroy model or logical. Logical block allows more optimizations. Concrete memory blocks support integer-pointer casts. To allow for as much optimizations as possible, memory blocks should be logical when allocated, and only be made concrete before an integer-pointer cast. This transformation is done with a new external function, the "capture" of a memory block.

\section{Mixed Simulation}
CompCert's correctness proof proves that the behavior of the target program is one of the behaviors of the source program. 
To do so, CompCert proves the backward simulation between the source program and the target program. 
To construct this simulation, CompCert used to prove forward simulation for each pass. This kinds of proof technique exploit determinism of target program to build backward simulation. However, CompCert using the quasi-concrete model is no longer deterministic, because the capture function is non-deterministic. We can solve this problem by replacing forward simulation with mixed simulation. We present the definition of mixed simulations, show that we can prove mixed simulations for every CompCert pass. \\
% We recommend abbrvnat bibliography style.

\bibliographystyle{abbrvnat}

% The bibliography should be embedded for final submission.

\bibliography{references}
 [1] Jeehoon Kang, Chung-Kil Hur, William Mansky, Dmitri Garbuzov, Steve Zdancewic, and Viktor Vafeiadis. A formal C memory model supporting integer-pointer casts. In Proceedings of the 36th ACM SIGPLAN Conference on Programming Language Design and Implementation, Portland, OR, USA, June 15-17, 2015, pages 326335, 2015.
\end{document}

%                       Revision History
%                       -------- -------
%  Date         Person  Ver.    Change
%  ----         ------  ----    ------

%  2013.06.29   TU      0.1--4  comments on permission/copyright notices

